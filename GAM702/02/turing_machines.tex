\part{Turing machines}
\frame{\partpage}

\begin{frame}{Turing machines}
    \begin{itemize}
        \pause\item Introduced in 1936 by Alan Turing
        \pause\item Theoretical model of a ``computer''
            \begin{itemize}
                \pause\item I.e.\ a machine that carries out computations (calculations)
            \end{itemize}
    \end{itemize}
\end{frame}

\begin{frame}{Turing machine}
    \begin{itemize}
        \pause\item Has a finite number of \textbf{states}
        \pause\item Has an infinite \textbf{tape}
        \pause\item Each space on the tape holds a \textbf{symbol} from a finite \textbf{alphabet}
        \pause\item Has a \textbf{tape head} pointing at one space on the tape
        \pause\item Has a transition table which, given:
            \begin{itemize}
                \item The current state
                \item The symbol under the tape head
            \end{itemize}
        specifies:
            \begin{itemize}
                \item A new state
                \item A new symbol to write to the tape, overwriting the current symbol
                \item Where to move the tape head: one space to the left, or one space to the right
            \end{itemize}
    \end{itemize}
\end{frame}

\newcommand{\stateA}{Drumstick}
\newcommand{\stateB}{Fruit}
\newcommand{\stateC}{Swizzels}
\newcommand{\tapeX}{Blank}
\newcommand{\tapeO}{Milk}
\newcommand{\tapeI}{White}

\begin{frame}{Activity}
    \begin{itemize}
        \pause\item In groups of 3-4
        \pause\item Line up 5-10 chocolates of different colours --- this is your \textbf{tape}
        \pause\item Point your \textbf{\stateA} lolly at the \textbf{leftmost} chocolate
            \begin{itemize}
                \pause\item The lolly is your \textbf{tape head}, and the type of lolly is your \textbf{state}
            \end{itemize}
        \pause\item Repeatedly apply the rules on the next slide
        \pause\item What computation does this machine perform?
            \begin{itemize}
                \pause\item Hint: $\text{\tapeO}=0$, $\text{\tapeI}=1$, and remember yesterday's lecture...
            \end{itemize}
    \end{itemize}
\end{frame}

\begin{frame}
    \begin{tabular}{|cc|ccc|} \hline
        Current & Current & New & New & Move \\
        lolly & chocolate & lolly & chocolate & direction \\\hline
        \stateA & \tapeX & \stateB & \tapeX & $\leftarrow$  \\
        \stateA & \tapeO & \stateA & \tapeI & $\rightarrow$ \\
        \stateA & \tapeI & \stateA & \tapeO & $\rightarrow$ \\\hline
        \stateB & \tapeX & \stateC & \tapeI & $\rightarrow$ \\
        \stateB & \tapeO & \stateC & \tapeI & $\leftarrow$  \\
        \stateB & \tapeI & \stateB & \tapeO & $\leftarrow$  \\\hline
        \stateC & \tapeX & Stop    & \tapeX & $\rightarrow$ \\
        \stateC & \tapeO & \stateC & \tapeO & $\leftarrow$  \\
        \stateC & \tapeI & \stateC & \tapeI & $\leftarrow$  \\\hline
    \end{tabular}
\end{frame}

\begin{frame}{The Church-Turing Thesis}
    \begin{itemize}
        \pause\item If a calculation can be carried out by a mechanical process at all,
            then it can be carried out by a Turing machine
        \pause\item I.e.\ a Turing machine is the most ``powerful'' computer possible,
            in terms of what is possible or impossible to compute
        \pause\item A machine, language or system is \textbf{Turing complete} if it can simulate a Turing machine
    \end{itemize}
\end{frame}

