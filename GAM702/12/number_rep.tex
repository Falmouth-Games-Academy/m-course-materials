\part{Representing numbers}
\frame{\partpage}

\lstset{language=Python}

\begin{frame}{Powers of 10}
	\pause
	$$ 10^6 = 1\underbrace{000000}_{\text{6 zeroes}} $$
	\pause
	$$ 10^1 = 10 $$
	\pause
	$$ 10^0 = 1 $$
	\pause
	$$ 10^{-1} = 0.1 $$
	\pause
	$$ 10^{-6} = 0.\underbrace{00000}_{\text{5 zeroes}}1 $$
\end{frame}

\begin{frame}{Scientific notation}
	\begin{itemize}
		\pause\item A way of writing \textbf{very large} and \textbf{very small} numbers
		\pause\item $a \times 10^b$, where
			\begin{itemize}
				\pause\item $a$ ($1 \leq |a| < 10$) is the \textbf{mantissa}
				\pause\item ($a$ is a positive or negative number
					with a single non-zero digit before the decimal point)
				\pause\item $b$ (an integer) is the \textbf{exponent}
			\end{itemize}
		\pause\item E.g.\ 1 light year = $9.461 \times 10^{15}$ metres
		\pause\item E.g.\ Planck's constant = $6.626 \times 10^{-34}$ joules
		\pause\item Socrative \texttt{FALCOMPED}
	\end{itemize}
\end{frame}

\begin{frame}[fragile]{Scientific notation in code}
	\pause Instead of writing \fbox{$\times 10$}, write \fbox{\lstinline{e}} (no spaces)
	\pause
	\begin{lstlisting}
lightYear = 9.461e15
plancksConstant = 6.626e-34
	\end{lstlisting}
\end{frame}

\begin{frame}{Floating point numbers}
	\begin{itemize}
		\pause\item Similar to scientific notation, but \textbf{base 2} (binary)
		\pause\item $\pm \text{mantissa} \times 2^{\text{exponent}}$
		\pause\item Sign is stored as a single bit: $0=+$, $1=-$
		\pause\item Mantissa is a binary number with a 1 before the point;
			only the digits after the point are stored
		\pause\item Exponent is a signed integer, stored with a \textbf{bias}
	\end{itemize}
\end{frame}

\begin{frame}[fragile]{IEEE 754 floating point formats}
	\begin{center}
		\begin{tabular}{|r|ccc|l|}
			\hline
			Type & Sign & Exponent & Mantissa & Total \\\hline
			Single precision & 1 bit & 8 bits & 23 bits & 32 bits \\\hline
			Double precision & 1 bit & 11 bits & 52 bits & 64 bits \\\hline
		\end{tabular}
	\end{center}
	\pause
	Exponent is stored with a \textbf{bias}:
	\begin{itemize}
		\item Single precision: store $\text{exponent} + 127$
		\item Double precision: store $\text{exponent} + 1023$
	\end{itemize}
	\begin{itemize}
		\pause\item Python uses double precision
		\pause\item Other languages have \lstinline[language=C]{float} (single) and \lstinline[language=C]{double} types
	\end{itemize}
\end{frame}

\begin{frame}{Example}
	\pause
	\begin{center}
		\texttt{0 10000001 10100000000000000000000}
	\end{center}
	\begin{itemize}
		\pause\item Exponent: $129 - 127 = 2$
		\pause\item Mantissa: binary $1.101$
		\pause\item $1 + \frac12 + \frac18 = 1.625$
		\pause\item $1.625 \times 2^2 = 6.5$
		\pause\item Alternatively: $1.101 \times 2^2 = 110.1$
		\pause\item $= 4 + 2 + \frac12 = 6.5$
	\end{itemize}
\end{frame}

\begin{frame}{Socrative \texttt{FALCOMPED}}
	\pause
	What is the value of this number expressed in IEEE 754 single precision format?
	\begin{center}
		\texttt{0 01111100 10011000000000000000000}
	\end{center}
	You have \textbf{5 minutes}, and you \textbf{may} use a calculator!
	(Unless your calculator does IEEE 754 conversion...)
\end{frame}

\begin{frame}{Precision of floating point numbers}
	\begin{itemize}
		\pause\item Precision \textbf{varies} by \textbf{magnitude}
		\pause\item Numbers near 0 can be stored more accurately than numbers further from 0
		\pause\item Analogy: in scientific notation with 3 decimal places
			\begin{itemize}
				\pause\item Around $3.142 \times 10^0$: can represent a difference of $0.001$
				\pause\item Around $3.142 \times 10^3$: can represent a difference of $1$
				\pause\item Around $3.142 \times 10^6$: can represent a difference of $1000$
			\end{itemize}
	\end{itemize}
\end{frame}

\begin{frame}[fragile]{Range of floating point numbers}
	\begin{center}
		\begin{tabular}{|r|cc|}
			\hline
			Type & Smallest value & Largest value \\\hline
			Single precision & $\pm 1.175 \times 10^{-38}$ & $\pm 3.403 \times 10^{38}$ \\\hline
			Double precision & $\pm 2.225 \times 10^{-308}$ & $\pm 1.798 \times 10^{308}$ \\\hline
		\end{tabular}
	\end{center}
\end{frame}

\begin{frame}{Rounding errors}
	\begin{itemize}
		\pause\item Many numbers cannot be represented exactly in IEEE float
			\begin{itemize}
				\pause\item Similar to how decimal notation cannot exactly represent
					$\frac13 = 0.3333333\dots$ or $\frac17 = 0.142857\dots$
			\end{itemize}
		\pause\item Decimal: can represent $\frac{a}{b}$ exactly iff $b = 2^m 5^n$
		\pause\item Binary: can represent $\frac{a}{b}$ exactly iff $b = 2^n$
		\pause\item In particular, IEEE float can't represent $\frac{1}{10} = 0.1$ exactly!
		\pause\item This can lead to \textbf{rounding errors} with some calculations
			\begin{itemize}
				\pause\item E.g.\ according to Python,
					$0.1 + 0.2 - 0.3 = 5.551 \times 10^{-17}$
			\end{itemize}
	\end{itemize}
\end{frame}

\begin{frame}[fragile]{Testing for equality}
	\begin{itemize}
		\pause\item Due to rounding errors, using \lstinline{==} or \lstinline{!=} with floating point numbers is almost always a bad idea
		\pause\item E.g.\ in Python, \lstinline[language=Python]{0.1 + 0.2 == 0.3} evaluates to \lstinline[language=Python]{False}
		\pause\item Better to check for \textbf{approximate equality}: calculate the difference between the numbers,
			and check that it's smaller than some threshold
		\pause
		\begin{lstlisting}
THRESHOLD = 1e-5
def is_approx_equal(a, b):
    return abs(b - a) < THRESHOLD
		\end{lstlisting}
	\end{itemize}
\end{frame}

\begin{frame}{Decimal types}
	\begin{itemize}
		\pause\item Python (and other languages) provide a \lstinline{decimal} type
		\pause\item Uses base 10 rather than base 2, so avoids some of the gotchas with IEEE float
		\pause\item ... however not natively supported by the CPU, hence much slower
	\end{itemize}
\end{frame}
