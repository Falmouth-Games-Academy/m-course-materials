\part{Game theory}
\frame{\partpage}

\begin{frame}{Game theory}
	\begin{itemize}
		\pause\item A branch of mathematics studying \textbf{decision making}
		\pause\item A \textbf{game} is a system where one or more \textbf{players} choose \textbf{actions};
			the combination of these choices lead to each agent receiving a \textbf{payoff}
		\pause\item Important applications in economics, ecology and social sciences as well as AI
	\end{itemize}
\end{frame}

\begin{frame}{The \sout{Prisoner's} Student's Dilemma}
	\begin{itemize}
		\pause\item Two students, \textbf{Alice} and \textbf{Bob}, are suspected of copying from each other
		\pause\item Each is offered a deal in exchange for information
		\pause\item Each can choose to \textbf{betray} the other or stay \textbf{silent}
			--- but they \textbf{cannot communicate} before deciding what to do
		\pause\item If \textbf{both stay silent}, both receive a C grade
		\pause\item If \textbf{Alice betrays Bob}, she receives an A whilst he gets expelled
		\pause\item If \textbf{Bob betrays Alice}, he receives an A whilst she gets expelled
		\pause\item If \textbf{both betray each other}, both get an F
	\end{itemize}
\end{frame}

\begin{frame}{Payoff matrix}
	\pause
	\begin{center}
		\begin{tabular}{|c|c|c|}
			\hline
			         & A silent & A betray \\\hline
			B silent & A: 50    & A: 70 \\
			         & B: 50    & B: -100 \\\hline
			B betray & A: -100  & A: 0 \\
			         & B: 70    & B: 0 \\\hline
		\end{tabular}
	\end{center}
\end{frame}

\iftoggle{printable}{}{
	\begin{frame}{Alice's thought process}
		\begin{itemize}
			\pause\item The best outcome overall is for both of us to stay silent
			\pause\item However if Bob stays silent, I can get a better mark by betraying him
			\pause\item And if Bob betrays me, I should betray him to avoid getting expelled
			\pause\item Therefore the rational choice is to betray
		\end{itemize}
		\pause ... and Bob's thought process is the same!
	\end{frame}
}

\begin{frame}{Nash equilibrium}
	\begin{itemize}
		\pause\item Consider the situation where both have chosen to betray
		\pause\item Neither person has anything to gain by switching to silence,
			assuming the other person doesn't also switch
		\pause\item Such a situation is called a \textbf{Nash equilibrium}
		\pause\item If all players are \textbf{rational} (in the sense of wanting to maximising payoff),
			they should converge upon a Nash equilibrium
	\end{itemize}
\end{frame}

\begin{frame}{Does every game have a Nash equilibrium?}
	\pause
	\begin{center}
		\begin{tabular}{|c|c|c|c|}
			\hline
			           & A rock & A paper & A scissors \\\hline
			B rock     & A: 0   & A: +1   & A: -1 \\
			           & B: 0   & B: -1   & B: +1 \\\hline
			B paper    & A: -1  & A: 0    & A: +1 \\
			           & B: +1  & B: 0    & B: -1 \\\hline
			B scissors & A: +1  & A: -1   & A: 0 \\
			           & B: -1  & B: +1   & B: 0 \\\hline
		\end{tabular}
	\end{center}
\end{frame}

\begin{frame}{Nash equilibrium for Rock-Paper-Scissors}
	\begin{itemize}
		\pause\item Committing to any choice of action can be \textbf{exploited}
		\pause\item E.g.\ if you always choose paper, I choose scissors
		\pause\item If we try to reason na\"ively, we get stuck in a loop
			\begin{itemize}
				\pause\item If I choose paper, you'll choose scissors, so I should choose rock, but then you'll choose paper,
					so I'll choose scissors, so you'll choose rock, so I choose paper...
			\end{itemize}
		\pause\item The optimum strategy is to be \textbf{unpredictable}
		\pause\item Choose rock with probability $\frac13$, paper with probability $\frac13$,
			scissors with probability $\frac13$
	\end{itemize}
\end{frame}

\begin{frame}{Mixed strategies}
	\begin{itemize}
		\pause\item A \textbf{mixed strategy} assigns probabilities to actions and chooses one at random
		\pause\item In contrast to a \textbf{pure} or \textbf{deterministic strategy}, which always chooses the same action
		\pause\item If we allow mixed strategies, \textbf{every game has at least one Nash equilibrium}
	\end{itemize}
\end{frame}

\begin{frame}{Guess $\frac23$ of the average}
	\begin{itemize}
		\pause\item Everyone guesses a real number (decimals are allowed) between 0 and 100 inclusive
		\pause\item The winner is the person who guesses closest to $\frac23$ of the mean of all guesses
		\pause\item Example:
			\begin{itemize}
				\pause\item If the guesses are 30, 40 and 80...
				\pause\item ... then the mean is $\frac{30+40+80}{3} = 50$...
				\pause\item ... so the winning guess is 30, as this is closest to $\frac23 \times 50 = 33.333$
			\end{itemize}
	\end{itemize}
\end{frame}

\iftoggle{printable}{}{
	\begin{frame}{The rational guess}
		\begin{itemize}
			\pause\item The average can't possibly be greater than 100
			\pause\item So no rational player would guess a number greater than 66.666
			\pause\item Which means the average can't possibly be greater than 66.666
			\pause\item So no rational player would guess greater than 44.444
			\pause\item Which means the average can't possibly be greater than 44.444
			\pause\item So no rational player would guess greater than 29.629
			\pause\item ... and so on ad infinitum
			\pause\item So the only \textbf{rational} guess is 0, as every rational player should guess 0 and $\frac23$ of 0 is 0
		\end{itemize}
	\end{frame}
}

\begin{frame}{Rationality}
	\begin{itemize}
		\pause\item Rationality is a useful assumption for mathematics and AI programmers
		\pause\item However it's important to remember that \textbf{humans aren't always rational}
	\end{itemize}
\end{frame}
