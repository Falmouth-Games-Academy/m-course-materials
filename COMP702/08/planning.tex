\part{Planning}
\frame{\partpage}

\begin{frame}{Planning}
	\begin{itemize}
		\pause\item An \textbf{agent} in an \textbf{environment}
		\pause\item The environment has a \textbf{state}
		\pause\item The agent can perform \textbf{actions} to change the state
		\pause\item Actions have a \textbf{cost} associated with them
		\pause\item The agent wants to change the state so as to achieve a \textbf{goal}
		\pause\item Problem: find a low-cost sequence of actions that leads to the goal
	\end{itemize}
\end{frame}

\begin{frame}{Planning as search}
	\begin{itemize}
		\pause\item We can construct a \textbf{state-action graph}
		\pause\item (Similar to a \textbf{game tree}, but may include \textbf{multiple paths} or \textbf{cycles})
		\pause\item Now the planning problem becomes very similar to the pathfinding problem (albeit possibly with multiple goals)
		\pause\item We can use many of the same algorithms (DFS, BFS, Dijkstra)
		\pause\item We can also use A$^*$ if we can come up with an admissible heuristic
	\end{itemize}
\end{frame}

\begin{frame}{Representing planning problems}
	\begin{itemize}
		\pause\item We can code the state-action representation manually
		\pause\item Or we can use a more general representation...
	\end{itemize}
\end{frame}

\part{STRIPS}
\frame{\partpage}

\begin{frame}{STRIPS planning}
	\begin{itemize}
		\pause\item \textbf{St}anford \textbf{R}esearch \textbf{I}nstitute \textbf{P}roblem \textbf{S}olver
		\pause\item Describes the state of the environment by a set of \textbf{predicates} which are true
		\pause\item (A predicate is basically a function which returns a \texttt{bool})
		\pause\item Models a problem as:
			\begin{itemize}
				\pause\item The \textbf{initial state} (a set of predicates which are true)
				\pause\item The \textbf{goal state} (a set of predicates, specifying whether each should be true or false)
				\pause\item The set of \textbf{actions}, each specifying:
					\begin{itemize}
						\pause\item Preconditions (a set of predicates which must be satisfied for this action to be possible) 
						\pause\item Postconditions (specifying what predicates are made true or false by this action)
					\end{itemize}
			\end{itemize}
	\end{itemize}
\end{frame}

\newcommand{\emptyy}{\includegraphics[height=0.1\textheight]{empty}}
\newcommand{\monkey}{\includegraphics[height=0.1\textheight]{monkey}}
\newcommand{\banana}{\includegraphics[height=0.1\textheight]{banana}}
\newcommand{\boxbox}{\includegraphics[height=0.1\textheight]{crate}}

\lstset{language={},
	basicstyle=\scriptsize\ttfamily
}

\begin{frame}{STRIPS example}
\end{frame}

\begin{frame}{STRIPS framework}
	\begin{itemize}
		\pause\item STRIPS gives a common framework for defining planning problems
		\pause\item Definitions in terms of \textbf{propositional logic}
		\pause\item Easy to \textbf{enumerate} and \textbf{simulate} actions, and hence search the state-action graph
		\pause\item Possible to write general-purpose STRIPS solvers
	\end{itemize}
\end{frame}

