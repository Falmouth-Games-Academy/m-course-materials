\part{Installing Python}
\frame{\partpage}

\begin{frame}{Getting Python for Windows}
    \begin{itemize}
        \pause\item Go to \url{https://www.python.org/}
        \pause\item Download the Windows x64 installer
        \pause\item Make sure to enable ``Add Python to PATH'' when installing
    \end{itemize}
\end{frame}

\begin{frame}{Getting Python for other OSes}
    \pause Mac OSX:
    \begin{itemize}
        \pause\item Comes with an outdated version of Python
        \pause\item Download the latest from \url{https://www.python.org/}, or install using Homebrew
    \end{itemize}
    \pause Linux:
    \begin{itemize}
        \pause\item May come preinstalled
        \pause\item If not, check the package manager for your distribution
    \end{itemize}
\end{frame}

\begin{frame}{Choosing an IDE}
    \begin{itemize}
        \pause\item \textbf{IDLE}
        \begin{itemize}
            \pause\item Comes with Python
            \pause\item Basic functionality
        \end{itemize}
        \pause\item \textbf{Visual Studio Code}
        \begin{itemize}
            \pause\item Install the Python extension
            \pause\item Useful features: syntax higlighting, autocomplete, linting, basic debugging
        \end{itemize}
        \pause\item \textbf{PyCharm}
        \begin{itemize}
            \pause\item \url{https://www.jetbrains.com/pycharm/}
            \pause\item Sign up for an educational account at \url{https://www.jetbrains.com/community/education/} to get the Professional version for free
            \pause\item Fully featured: advanced debugging, package management
        \end{itemize}
    \end{itemize}
\end{frame}

\begin{frame}{Three ways to run Python code}
    \begin{itemize}
        \pause\item Write code in a \texttt{.py} file and run it
        \pause\item Type code into the interactive interpreter
        \pause\item Use a more advanced interactive environment e.g.\ Jupyter Notebook
    \end{itemize}
\end{frame}

\begin{frame}{Package management}
    \begin{itemize}
        \pause\item Python has many useful \textbf{packages} available
        \pause\item Can be installed using \textbf{pip}
        \pause\item From the command line: \texttt{pip install ...}
        \pause\item In PyCharm: File $\to$ Settings $\to$ Project $\to$ Python Interpreter
    \end{itemize}
\end{frame}

