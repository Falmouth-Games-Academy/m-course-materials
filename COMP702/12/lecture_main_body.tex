% !TeX root = ./2020-21-COMP702-12-lecture-materials.tex
% Adjust these for the path of the theme and its graphics, relative to this file
%\usepackage{beamerthemeFalmouthGamesAcademy}
\usepackage{../../beamerthemeFalmouthGamesAcademy}
\usepackage{multimedia}
\graphicspath{ {../../} }

% Default language for code listings
\lstset{language=Python
}

% For strikethrough effect
\usepackage[normalem]{ulem}
\usepackage{wasysym}

\usepackage{pdfpages}
\usepackage{algpseudocode}

% http://www.texample.net/tikz/examples/state-machine/
\usetikzlibrary{arrows,automata}

\newcommand{\modulecode}{COMP702}\newcommand{\moduletitle}{Classical Artificial Intelligence}\newcommand{\sessionnumber}{1}

\hypersetup{
pdftex,
pdftitle=\sessionnumber: Introduction to Python,
pdfauthor=Ed Powley,
pdfdisplaydoctitle,
pdflang=en-GB
}

\begin{document}
\title{\sessionnumber: Introduction to Python}
\subtitle{\modulecode: \moduletitle}

\frame{\titlepage} 

\part{Installing Python}
\frame{\partpage}

\begin{frame}{Getting Python for Windows}
    \begin{itemize}
        \pause\item Go to \url{https://www.python.org/}
        \pause\item Download the Windows x64 installer
        \pause\item Make sure to enable ``Add Python to PATH'' when installing
    \end{itemize}
\end{frame}

\begin{frame}{Getting Python for other OSes}
    \pause Mac OSX:
    \begin{itemize}
        \pause\item Comes with an outdated version of Python
        \pause\item Download the latest from \url{https://www.python.org/}, or install using Homebrew
    \end{itemize}
    \pause Linux:
    \begin{itemize}
        \pause\item May come preinstalled
        \pause\item If not, check the package manager for your distribution
    \end{itemize}
\end{frame}

\begin{frame}{Choosing an IDE}
    \begin{itemize}
        \pause\item \textbf{IDLE}
        \begin{itemize}
            \pause\item Comes with Python
            \pause\item Basic functionality
        \end{itemize}
        \pause\item \textbf{Visual Studio Code}
        \begin{itemize}
            \pause\item Install the Python extension
            \pause\item Useful features: syntax higlighting, autocomplete, linting, basic debugging
        \end{itemize}
        \pause\item \textbf{PyCharm}
        \begin{itemize}
            \pause\item \url{https://www.jetbrains.com/pycharm/}
            \pause\item Sign up for an educational account at \url{https://www.jetbrains.com/community/education/} to get the Professional version for free
            \pause\item Fully featured: advanced debugging, package management
        \end{itemize}
    \end{itemize}
\end{frame}

\begin{frame}{Three ways to run Python code}
    \begin{itemize}
        \pause\item Write code in a \texttt{.py} file and run it
        \pause\item Type code into the interactive interpreter
        \pause\item Use a more advanced interactive environment e.g.\ Jupyter Notebook
    \end{itemize}
\end{frame}

\begin{frame}{Package management}
    \begin{itemize}
        \pause\item Python has many useful \textbf{packages} available
        \pause\item Can be installed using \textbf{pip}
        \pause\item From the command line: \texttt{pip install ...}
        \pause\item In PyCharm: File $\to$ Settings $\to$ Project $\to$ Python Interpreter
    \end{itemize}
\end{frame}


\part{Python for C\# programmers}
\frame{\partpage}

\begin{frame}[fragile]{Hello World}
    \begin{lstlisting}
print("Hello, world!")
    \end{lstlisting}
    \begin{itemize}
        \pause\item Code does not have to be inside a class or function
        \pause\item No semicolons
        \pause\item \lstinline{print} is a built-in function
    \end{itemize}
\end{frame}

\begin{frame}[fragile]{Comments}
    \begin{lstlisting}
# This is a comment
    \end{lstlisting}
\end{frame}

\begin{frame}[fragile]{Variables}
    \begin{lstlisting}
a = 1
b = 2
c = a + b
    \end{lstlisting}
    \begin{itemize}
        \pause\item Variables do not need to be declared
    \end{itemize}
\end{frame}

\begin{frame}[fragile]{Variables}
    \begin{lstlisting}
x = 7
x = "Hello"
    \end{lstlisting}
    \begin{itemize}
        \pause\item Variables can hold values of any type
    \end{itemize}
\end{frame}

\begin{frame}[fragile]{If statement}
    \begin{lstlisting}
if x < 10:
    print("asdf")
elif x < 20:
    print("qwerty")
else:
    print("zxcv")
    \end{lstlisting}
    \begin{itemize}
        \pause\item Indentation matters
        \pause\item Note the colons and the lack of parentheses
    \end{itemize}
\end{frame}

\begin{frame}[fragile]{Lists}
    \begin{lstlisting}
my_list = ["Hello", "World", "Foo", 42]
print(my_list[0])
    \end{lstlisting}
    \begin{itemize}
        \pause\item Can store values of any type
    \end{itemize}
\end{frame}

\begin{frame}[fragile]{For loop}
    \begin{lstlisting}
for x in my_list:
    print(x)
    \end{lstlisting}
    \begin{itemize}
        \pause\item Works like \lstinline{foreach} in C\#
    \end{itemize}
\end{frame}

\begin{frame}[fragile]{For loop}
    \begin{lstlisting}
for i in range(10):
    print(i)
    \end{lstlisting}
    \begin{itemize}
        \pause\item Python doesn't have C-style \lstinline{for} loops
        \pause\item Built-in function \lstinline{range(n)} gives numbers from $0$ to $n-1$
    \end{itemize}
\end{frame}

\begin{frame}[fragile]{Functions}
    \begin{lstlisting}
def add(a, b):
    return a + b
    \end{lstlisting}
    \begin{itemize}
        \pause\item Can return any value, or nothing
    \end{itemize}
\end{frame}

\begin{frame}[fragile]{Functions are values}
    \begin{lstlisting}
def add(a, b):
    return a + b

x = add

print(x(3, 4))
    \end{lstlisting}
\end{frame}

\begin{frame}[fragile]{Classes}
    \begin{lstlisting}
class Thing:
    def __init__(self, a, b):
        self.a = a
        self.b = b

    def add(self):
        return self.a + self.b

x = Thing(2, 3)
    \end{lstlisting}
    \begin{itemize}
        \pause\item \lstinline{__init__} is the constructor
        \pause\item \lstinline{self} is equivalent to \lstinline{this}
        \pause\item \lstinline{self} is never implicit, unlike \lstinline{this}
    \end{itemize}
\end{frame}

\begin{frame}[fragile]{List comprehensions}
    \begin{lstlisting}
my_list = [1, 3, 6, 10]
my_other_list = [x*2 for x in my_list if x < 10]
    \end{lstlisting}
    \begin{itemize}
        \pause\item Similar to LINQ queries in C\#
    \end{itemize}
\end{frame}

\begin{frame}{Python and C}
    \begin{itemize}
        \pause\item Python has many advantages, but speed is not one of them...
        \pause\item For intensive calculations we generally rely on external libraries written in C/C++
        \pause\item It is also possible to embed the Python interpreter in a C/C++ program
    \end{itemize}
\end{frame}

\begin{frame}{Useful libraries}
    \begin{itemize}
        \pause\item \textbf{NumPy}: fast calculation with $N$-dimensional numerical arrays
        \pause\item \textbf{SciPy}: various scientific tools, including statistical analysis
        \pause\item \textbf{Pandas}: importing and manipulation of large datasets
        \pause\item \textbf{Matplotlib}: plotting of charts and graphs
        \pause\item Various libraries for machine learning, which you will learn about in COMP704...
    \end{itemize}
\end{frame}

\part{Pygame}
\frame{\partpage}

\begin{frame}{Pygame}
    \begin{itemize}
        \pause\item 2D game development library
        \pause\item Based on SDL (Simple DirectMedia Layer)
        \pause\item Allows low-level control over the game loop
    \end{itemize}
\end{frame}

\begin{frame}{The main game loop}
    \pause Most main loops in games follow the same basic pattern:
    \pause
    \begin{algorithmic}
        \Repeat
            \pause \State handle events
            \pause \State update game state
            \pause \State draw graphics
            \pause \State sleep to maintain frame rate
            \pause \State swap buffers
        \pause \Until{``quit'' event received}
    \end{algorithmic}
\end{frame}

\begin{frame}{Double buffering}
    \begin{itemize}
        \pause\item Pygame (and other frameworks) use \textbf{two} graphics buffers
        \pause\item One is displayed on the screen
        \pause\item The other is used to draw the next frame
        \pause\item When drawing is finished, they are \textbf{swapped} --- the drawn buffer is displayed, and the old buffer is used to draw the next frame
    \end{itemize}
\end{frame}


\part{Workshop}
\frame{\partpage}

\begin{frame}{Workshop}
    \begin{itemize}
        \item Begin working through the Python resources linked on LearningSpace
        \item Suggested Christmas project: use Pygame to reimplement your favourite 80s arcade game!
    \end{itemize}
\end{frame}

\end{document}
